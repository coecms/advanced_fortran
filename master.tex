\documentclass{beamer}
\usepackage{listings}
\usepackage{color}

\definecolor{dkgreen}{rgb}{0,0.6,0}
\definecolor{gray}{rgb}{0.5,0.5,0.5}
\definecolor{mauve}{rgb}{0.58,0,0.82}

\lstset{frame=single,
  language=Fortran,
  aboveskip=3mm,
  belowskip=3mm,
  showstringspaces=false,
  columns=flexible,
  basicstyle={\small\ttfamily},
  numbers=none, %left,
  numberstyle=\tiny\color{gray},
  keywordstyle=\color{blue},
  commentstyle=\color{dkgreen},
  stringstyle=\color{mauve},
  breaklines=true,
  breakatwhitespace=true,
  tabsize=3
}

\usetheme{Berlin}

\title{Advanced Fortran}
\author{Holger Wolff}
\date{30/05/2016}


\begin{document}

\begin{frame}
  \titlepage
\end{frame}

\section*{}

\begin{frame}
  \tableofcontents
\end{frame}

\begin{frame}
    \frametitle{Opening Statements}
    Half a year ago, Melissa asked me to do an `Introduction into Fortran'
    session for the students. 

    Introduction is easy: Give the people some pointers, some starting hints, and then set them free.

    Now she asked me to do an `Advanced Fortran' -- That's a lot harder.
    So I deflected.

    Instead of talking about Fortran, I'll talk about common programming practices with Fortran as the application.
\end{frame}

\section{Compiler}
\begin{frame}
    \frametitle{Compiler}
    The main Fortran Compilers used in this centre are:
    \begin{itemize}
        \item Intel Fortran Compiler
        \item gfortran
    \end{itemize}
    Intel Fortran Compiler is better, gfortran is part of the GNU Compiler Collection and is free.

    On NCI, \texttt{mpif90} and \texttt{mpifort} are used as wrappers for Intel Fortran. 
\end{frame}
\begin{frame}[fragile]
    \frametitle{Compiler Warnings}
    \begin{lstlisting}[language=bash]
    -warn all           (ifort)
    -Wall               (gfortran)
    \end{lstlisting}
    Warnings are unexpected behaviour, e.g. using of uninitialised variables, unreachable code.
\end{frame}
\begin{frame}[fragile]
    \frametitle{Checks}
    \begin{lstlisting}[language=bash]
    -check all          (ifort)    
    -fcheck=all         (gfortran)
    \end{lstlisting}
    Run time checks slow the execution down, but useful during development.
    Checks include array bounds, but also give run-time warnings when temporary arrays are created.
\end{frame}
\begin{frame}[fragile]
    \frametitle{Optimisation Levels}
    \begin{lstlisting}[language=bash]
    -O<n>
    \end{lstlisting}
    \texttt{n} between 0 and 3.
    \begin{itemize}
    \item \texttt{-O0} means no optimisation, everything exactly as written
    \item \texttt{-O2} normal optimisation, usually safe
    \item \texttt{-O3} strong optimisation, can introduce errors, use with care.
    \end{itemize}
\end{frame}
\begin{frame}[fragile]
    \frametitle{Debugging Symbols}
    \begin{lstlisting}[language=bash]
    -g
    \end{lstlisting}
    Include actual code in executable. Can be read by debuggers or \texttt{objdump -S}
\end{frame}
\begin{frame}[fragile]
    \frametitle{Includes}
    \begin{lstlisting}[language=bash]
    -I <CPATH>
    -i <MOD_FILE>
    \end{lstlisting}
    \texttt{-i} and \texttt{-I} are used for compilation. They give the file (\texttt{-i}) and directory (\texttt{-I}) of the include and \texttt{.mod} files.

    \begin{lstlisting}[numbers=none]
    -I ./includes -l foo
    \end{lstlisting}
    searches for the include file \texttt{foo.mod} in \texttt{./includes} first, then all directories in \texttt{\$CPATH}.
\end{frame}
\begin{frame}[fragile]
    \frametitle{Linking to Libraries}
    \begin{lstlisting}[language=bash]
    -L <LIB_PATH>
    -l <LIB_NAME>
    \end{lstlisting}
    \texttt{-l} and \texttt{-L} are used for linking. They give the directory (\texttt{-L}) and library name (\texttt{-l}). 

    \begin{lstlisting}[language=bash]
    -L ./libs -l foo
    \end{lstlisting}
    searches for the library files \texttt{libfoo.a} or \texttt{libfoo.so} in \texttt{./libs} first, then all directories in \texttt{\$LD\_LIBRARY\_PATH}.
\end{frame}

\begin{frame}
    \frametitle{Questions}
    This concludes this short chapter on compiler options.

    Any questions so far?
\end{frame}


\section{Makefile}
\begin{frame}
    \frametitle{Compilation Automation}
    Compiling larger projects can become tiresome.

    Compilation Automation has been developed some time ago. 
    Most successful system: \textbf{make}
\end{frame}

\defverbatim[colored]\Lst{%
\begin{lstlisting}[tabsize=8,showtabs,frame=single]
<target> : <sources>
	<commands>
\end{lstlisting}}

\begin{frame}
    \frametitle{Makefile}
    \textbf{make} gets its instructions from a specific file in the current directory:

    \texttt{Makefile}

    It contains rules that follow this syntax:
\Lst
    Note that the \texttt{<commands>} are indented by a \texttt{<TAB>} character, not spaces!
\end{frame}

\defverbatim[colored]\LstBasic{%
\begin{lstlisting}[tabsize=8,showtabs,frame=single]
foo : foo.f90
	ifort -o foo foo.f90
\end{lstlisting}}

\begin{frame}
    \frametitle{Basic}
    The most basic rule would look like this:
\LstBasic
\end{frame}

\defverbatim[colored]\LstRef{%
\begin{lstlisting}[tabsize=8,showtabs,frame=single]
foo : foo.f90
	ifort -o $@ $<
\end{lstlisting}}

\begin{frame}
    \frametitle{Basic}
    We can make some substitutions:
    \begin{itemize}
        \item \texttt{\$@} refers to the target.
        \item \texttt{\$<} refers to the first source
        \item \texttt{\$\^{ }} refers to all the sources
    \end{itemize}
\LstRef
\end{frame}

\defverbatim[colored]\LstGen{%
\begin{lstlisting}[tabsize=8,showtabs,frame=single]
%.o : %.f90
	ifort -c -o $@ $<

foo : foo.o bar.o
	ifort -o $@ $^
\end{lstlisting}}

\begin{frame}
    \frametitle{Automatic Rules}
    Often, the rule is identical for many if not all objects, so we can generalise:
\LstGen
\end{frame}

\defverbatim[colored]\LstVars{%
\begin{lstlisting}[tabsize=8,showtabs,frame=single]
FC = ifort
FFLAGS = -warn all -check all
LD = $(FC)
LDFLAGS = $(FFLAGS) -l foo

%.o : %.f90
	$(FC) $(FFLAGS) -o $@ $<

foo : foo.o bar.o
	$(LD) $(LDFLAGS) -o $@ $^
\end{lstlisting}}

\begin{frame}
    \frametitle{Automatic Rules}
    Variables can be used to keep simplity it further. In the example below, it would be much easier to change between compilers:
\LstVars
\end{frame}


\section{Coding Practices}

% \begin{frame}
%     \frametitle{Comments}
%     We all use comments too sparingly.
%     
%     Someone else, or you in 2 years, will not remember what was in your head when you wrote this piece of code.
%     So when writing code, keep readability in mind. 
% 
% \end{frame}
% 
% \begin{frame}[fragile]
%     \frametitle{Comments continued}
% 
%     Comments should convey why you did what you did, not what you did.
%     \begin{lstlisting}
%     ! increment i
%     i = i + 1
%     \end{lstlisting}
%     is useless
% 
%     \begin{lstlisting}
%     ! Calc absolute distance between A and B
%     dist = sqrt((A(1) - B(1))**2 + (A(2) - B(2))**2)
%     \end{lstlisting}
%     is good.
% 
% \end{frame}
% 
\begin{frame}
    \frametitle{Understandable Code}
    While writing code, you are actively thinking about writing this code.
    Every line that you write will make perfect sense while you're writing it.

    But when you look at someone else's code, or even your own code from 6 months ago,
    you won't have those same thoughts in your head any more.

    And then you will have to spend a lot of brain power to figure out what the code is doing,
    and why it's doing what it's doing.

    For your own sanity: \textbf{Keep your code understandable}
\end{frame}

\begin{frame}[fragile]
    \frametitle{Comments}
    The easiest way to keep your code understandable is to write comments.
    In general: The more the merrier, but:

    Comments should explain \textbf{why} you're doing what you're doing,
    rather than what you're doing:

    \begin{lstlisting}[language=Fortran]
    ! Increment i
    i = i + 1
    \end{lstlisting}
    This comment is not useful.
\end{frame}

\begin{frame}
    \frametitle{Naming}
    Often you will think up names. 
    Names for procedures, variables.

    Think about what to call it. 
    Make sure that the name really represents what is stored in the variable.

    It's okay to have an iterator called \texttt{i}, if you're consistent about it.
    
    But if you want to store some user name, please call it something like \texttt{cUserName}, and not just \texttt{u}.
    (The leading \texttt{c} stands for the type of variable: \texttt{CHARACTER}.)
\end{frame}

\begin{frame}[fragile]
    \frametitle{Magic Numbers}
    Sometimes you need some numbers in your code, for example to check whether another number is even:
    \begin{lstlisting}
    if (mod(n, 2) == 0) then
    \end{lstlisting}

    But sometimes it's more confusing:
    \begin{lstlisting}
    if (iErrorCode == 264) then
    \end{lstlisting}
    What's the \texttt{264}? What do I do here? This \texttt{264} is what programmers call a 'magic number' --
    and it hinders understanding the code.
\end{frame}

\begin{frame}[fragile]
    \frametitle{Avoiding Magic Numbers}
    The way to avoid magic numbers is to create constants.
    Say, this \texttt{264} was the error code for \textbf{File Not Found}.

    Then you could do something like this:
    \begin{lstlisting}
        INTEGER, PARAMETER :: FileNotFoundError = 264
        ...
        if (iErrorCode == FileNotFoundError) then
    \end{lstlisting}
    Far easier to read.
\end{frame}

\begin{frame}
    \frametitle{Error Checking/Handling}
    Many Fortran routines that could conceivably fail have 
    optional error arguments.

    Examples:

    \begin{itemize}
        \item \texttt{OPEN}, \texttt{READ}, \texttt{WRITE}, even \texttt{CLOSE} have an optional \texttt{IOSTAT} argument that returns an \texttt{INTEGER}.
        \item \texttt{ALLOCATE}, \texttt{DEALLOCATE} have \texttt{STAT}.
    \end{itemize}

    By convention, 0 means that there was no issue. A negative value refers to a warning, and a positive value corresponds to an error.
    Think about what might happen, and how the program should react to this.

    Use them.
\end{frame}

\begin{frame}[fragile]
    \frametitle{Error passing}
    Two main error passing philosophies:

    In one, the error is an (often optional) \texttt{INTENT(OUT)} parameter. 
    \begin{lstlisting}
    CALL mySub(arg1, arg2, error)
    \end{lstlisting}

    In the other, you call a function that returns the error code:
    \begin{lstlisting}
    error = myFunc(arg1, arg2)
    \end{lstlisting}

\end{frame}

\begin{frame}
    \frametitle{Fragile and Robust}
    Error handling allows you to design your software to be either fragile or robust:

    A \textbf{robust} program will attempt to plow ahead, possibly logging the error. 
    While desirable for a deployed program, this might obscure the real cause of odd behaviour later on.

    A \textbf{fragile} program will fall over early upon encountering an unexpected situation. 
    This is particularly useful during programming and debugging, but can be annoying in production.
\end{frame}


\section{Debugging}

\begin{frame}[fragile]
    \frametitle{Tracing}
    Tracing is the simplest form of evaluating the execution.

    Basically it means a lot of statements like

    \begin{lstlisting}
        print *, "Been here, value of X is ", X
    \end{lstlisting}

    While it is very easy to execute, it requires a lot of code changes and recompiles.

    Plus, at the end you have to remove all the \texttt{print} statements again.
\end{frame}

\begin{frame}
    \frametitle{Advanced Debugging}
    Debugging is an Art Form.

    There are several tools available to help, I'll showcase a few examples:

    \begin{itemize}
        \item Stack Trace
        \item Run Time Checks
        \item Debugger
    \end{itemize}

\end{frame}



\begin{frame}
    \frametitle{Testing}
    pfUnit 
\end{frame}



\end{document}

