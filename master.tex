\documentclass{beamer}
\usepackage{listings}
\usepackage{color}

\definecolor{dkgreen}{rgb}{0,0.6,0}
\definecolor{gray}{rgb}{0.5,0.5,0.5}
\definecolor{mauve}{rgb}{0.58,0,0.82}

\lstset{frame=single,
  language=Fortran,
  aboveskip=3mm,
  belowskip=3mm,
  showstringspaces=false,
  columns=flexible,
  basicstyle={\small\ttfamily},
  numbers=none, %left,
  numberstyle=\tiny\color{gray},
  keywordstyle=\color{blue},
  commentstyle=\color{dkgreen},
  stringstyle=\color{mauve},
  breaklines=true,
  breakatwhitespace=true,
  tabsize=3
}

\usetheme{Berlin}

\title{Advanced Fortran}
\author{Holger Wolff}
\date{30/05/2016}


\begin{document}

\begin{frame}
  \titlepage
\end{frame}

\section*{}

\begin{frame}
  \tableofcontents
\end{frame}

\begin{frame}
    \frametitle{Opening Statements}
    Half a year ago, Melissa asked me to do an `Introduction into Fortran'
    session for the students. 

    Introduction is easy: Give the people some pointers, some starting hints, and then set them free.

    Now she asked me to do an `Advanced Fortran' -- That's a lot harder.
    So I deflected.

    Instead of talking about Fortran, I'll talk about common programming practices with Fortran as the application.
\end{frame}

\section{Compiler}
\begin{frame}
    \frametitle{Compiler}
    The main Fortran Compilers used in this centre are:
    \begin{itemize}
        \item Intel Fortran Compiler
        \item gfortran
    \end{itemize}
    Intel Fortran Compiler is better, gfortran is part of the GNU Compiler Collection and is free.

    On NCI, \texttt{mpif90} and \texttt{mpifort} are used as wrappers for Intel Fortran. 
\end{frame}
\begin{frame}[fragile]
    \frametitle{Compiler Warnings}
    \begin{lstlisting}[language=bash]
    -warn all           (ifort)
    -Wall               (gfortran)
    \end{lstlisting}
    Warnings are unexpected behaviour, e.g. using of uninitialised variables, unreachable code.
\end{frame}
\begin{frame}[fragile]
    \frametitle{Checks}
    \begin{lstlisting}[language=bash]
    -check all          (ifort)    
    -fcheck=all         (gfortran)
    \end{lstlisting}
    Run time checks slow the execution down, but useful during development.
    Checks include array bounds, but also give run-time warnings when temporary arrays are created.
\end{frame}
\begin{frame}[fragile]
    \frametitle{Optimisation Levels}
    \begin{lstlisting}[language=bash]
    -O<n>
    \end{lstlisting}
    \texttt{n} between 0 and 3.
    \begin{itemize}
    \item \texttt{-O0} means no optimisation, everything exactly as written
    \item \texttt{-O2} normal optimisation, usually safe
    \item \texttt{-O3} strong optimisation, can introduce errors, use with care.
    \end{itemize}
\end{frame}
\begin{frame}[fragile]
    \frametitle{Debugging Symbols}
    \begin{lstlisting}[language=bash]
    -g
    \end{lstlisting}
    Include actual code in executable. Can be read by debuggers or \texttt{objdump -S}
\end{frame}
\begin{frame}[fragile]
    \frametitle{Includes}
    \begin{lstlisting}[language=bash]
    -I <CPATH>
    -i <MOD_FILE>
    \end{lstlisting}
    \texttt{-i} and \texttt{-I} are used for compilation. They give the file (\texttt{-i}) and directory (\texttt{-I}) of the include and \texttt{.mod} files.

    \begin{lstlisting}[numbers=none]
    -I ./includes -l foo
    \end{lstlisting}
    searches for the include file \texttt{foo.mod} in \texttt{./includes} first, then all directories in \texttt{\$CPATH}.
\end{frame}
\begin{frame}[fragile]
    \frametitle{Linking to Libraries}
    \begin{lstlisting}[language=bash]
    -L <LIB_PATH>
    -l <LIB_NAME>
    \end{lstlisting}
    \texttt{-l} and \texttt{-L} are used for linking. They give the directory (\texttt{-L}) and library name (\texttt{-l}). 

    \begin{lstlisting}[language=bash]
    -L ./libs -l foo
    \end{lstlisting}
    searches for the library files \texttt{libfoo.a} or \texttt{libfoo.so} in \texttt{./libs} first, then all directories in \texttt{\$LD\_LIBRARY\_PATH}.
\end{frame}

\begin{frame}
    \frametitle{Questions}
    This concludes this short chapter on compiler options.

    Any questions so far?
\end{frame}


\section{Makefile}
\begin{frame}
    \frametitle{Makefile}
\end{frame}


\section{Coding Practices}

% \begin{frame}
%     \frametitle{Comments}
%     We all use comments too sparingly.
%     
%     Someone else, or you in 2 years, will not remember what was in your head when you wrote this piece of code.
%     So when writing code, keep readability in mind. 
% 
% \end{frame}
% 
% \begin{frame}[fragile]
%     \frametitle{Comments continued}
% 
%     Comments should convey why you did what you did, not what you did.
%     \begin{lstlisting}
%     ! increment i
%     i = i + 1
%     \end{lstlisting}
%     is useless
% 
%     \begin{lstlisting}
%     ! Calc absolute distance between A and B
%     dist = sqrt((A(1) - B(1))**2 + (A(2) - B(2))**2)
%     \end{lstlisting}
%     is good.
% 
% \end{frame}
% 

\begin{frame}
    \frametitle{Error Checking/Handling}
    Many Fortran routines that could conceivably fail have 
    optional error arguments.

    Examples:

    \begin{itemize}
        \item \texttt{OPEN}, \texttt{READ}, \texttt{WRITE}, even \texttt{CLOSE} have an optional \texttt{IOSTAT} argument that returns an \texttt{INTEGER}.
        \item \texttt{ALLOCATE}, \texttt{DEALLOCATE} have \texttt{STAT}.
    \end{itemize}

    By convention, 0 means that there was no issue. A negative value refers to a warning, and a positive value corresponds to an error.
    Think about what might happen, and how the program should react to this.

    Use them.
\end{frame}

\begin{frame}[fragile]
    \frametitle{Error passing}
    Two main error passing philosophies:

    In one, the error is an (often optional) \texttt{INTENT(OUT)} parameter. 
    \begin{lstlisting}
    CALL mySub(arg1, arg2, error)
    \end{lstlisting}

    In the other, you call a function that returns the error code:
    \begin{lstlisting}
    error = myFunc(arg1, arg2)
    \end{lstlisting}

\end{frame}

\begin{frame}
    \frametitle{Fragile and Robust}
    Error handling allows you to design your software to be either fragile or robust:

    A \textbf{robust} program will attempt to plow ahead, possibly logging the error. 
    While desirable for a deployed program, this might obscure the real cause of odd behaviour later on.

    A \textbf{fragile} program will fall over early upon encountering an unexpected situation. 
    This is particularly useful during programming and debugging, but can be annoying in production.
\end{frame}


\section{Debugging}

\begin{frame}[fragile]
    \frametitle{Tracing}
    Tracing is the simplest form of evaluating the execution.

    Basically it means a lot of statements like

    \begin{lstlisting}
        print *, "Been here, value of X is ", X
    \end{lstlisting}

    While it is very easy to execute, it requires a lot of code changes and recompiles.

    Plus, at the end you have to remove all the \texttt{print} statements again.
\end{frame}

\begin{frame}
    \frametitle{Advanced Debugging}
    Debugging is an Art Form.

    There are several tools available to help, I'll showcase a few examples:

    \begin{itemize}
        \item Stack Trace
        \item Run Time Checks
        \item Debugger
    \end{itemize}

\end{frame}


\begin{frame}
    \frametitle{Questions}
    This concludes this short chapter on debugging.

    Any questions so far?
\end{frame}


\begin{frame}
    \frametitle{Testing}
    pfUnit 
\end{frame}



\end{document}

