\section{Unit Testing}

\begin{frame}
    \frametitle{Unit Testing Background}
    Testing individual functions/subroutines ("Units") independently.

    Helps to ensure that each part of a program is working as intended.

    Continued testing helps to determine wether changes break functionality.
\end{frame}

\begin{frame}
    \frametitle{xUnit}
    Most Unit Testing Frameworks follow the \textbf{xUnit} layout.

    \begin{itemize}
        \item Each test is similar to a routine call in the specific programming language.
        \item Keyword or routine name beginning with \texttt{test\_} to indicate tests
        \item \texttt{ASSERT} statement to evaluate each test.
        \item Optional \texttt{setup} and \texttt{teardown} procedures that are called
            before/after each test.
    \end{itemize}
\end{frame}

\begin{frame}[fragile]
    \frametitle{pFUnit}
    Uses Python to create wrapper Fortran code for each test.

    \begin{lstlisting}
    @test
    subroutine test_a_equals_a()
        use pfunit_mod
        implicit none
        @assertEqual("a", "a")
    end subroutine test_a_equals_a
    \end{lstlisting}
\end{frame}

\begin{frame}[fragile]
    \frametitle{pFUnit continued}
    \begin{itemize}
        \item Files containing tests have file extension \texttt{.pf}
        \item File called \texttt{testSuites.inc} containing the file names.
    \end{itemize}
    Example:
    If there are tests in a file called \texttt{test\_something.pf}, then there needs to be a line 
    \begin{lstlisting}
    ADD_TEST_SUITE(test_something_suite)
    \end{lstlisting}
    in the file \texttt{testSuites.inc}.
\end{frame}


\begin{frame}
    \frametitle{Test Driven Development}
    Writing Tests before actually implementing the functionality helps focus on
    how the unit is supposed to behave.

    TDD cycle:
    \begin{enumerate}
        \item{Write Test}
        \item{Verify Test fails}
        \item{Implement Test}
        \item{Verify Test succeeds}
    \end{enumerate}
\end{frame}

\begin{frame}
    \frametitle{TDD example}
    Example: Use TDD to build a greeter.
    \begin{itemize}
        \item If called without arguments, it should return ``Hello World''
        \item An optional \texttt{cGreeting} dummy argument can be used to replace the ``Hello''
        \item An optional \texttt{cAdressee} dummy argument can be used to replace the ``World''
    \end{itemize}
\end{frame}

\begin{frame}
    \frametitle{Questions}
    This concludes the prepared part of this tutorial.

    Any final questions?
\end{frame}
