\section{Build Automation}
\begin{frame}
    \frametitle{Build Automation}
    Compiling larger projects can become tiresome.

    Compilation Automation has been developed some time ago. 
    Most successful system: \textbf{make}
\end{frame}

\defverbatim[colored]\Lst{%
\begin{lstlisting}[tabsize=8,showtabs,frame=single,language=make]
<target> : <sources>
	<commands>
\end{lstlisting}}

\begin{frame}
    \frametitle{Makefile}
    \textbf{make} gets its instructions from a specific file in the current directory:

    \texttt{Makefile}

    It contains rules that follow this syntax:
\Lst
    Note that the \texttt{<commands>} are indented by a \texttt{<TAB>} character, not spaces!
\end{frame}

\defverbatim[colored]\LstBasic{%
\begin{lstlisting}[tabsize=8,showtabs,frame=single,language=make]
foo : foo.f90
	ifort -o foo foo.f90
\end{lstlisting}}

\begin{frame}
    \frametitle{Basic}
    The most basic rule would look like this:
\LstBasic
\end{frame}

\defverbatim[colored]\LstRef{%
\begin{lstlisting}[tabsize=8,showtabs,frame=single,language=make]
foo : foo.f90
	ifort -o $@ $<
\end{lstlisting}}

\begin{frame}
    \frametitle{Basic}
    We can make some substitutions:
    \begin{itemize}
        \item \texttt{\$@} refers to the target.
        \item \texttt{\$<} refers to the first source
        \item \texttt{\$\^{ }} refers to all the sources
    \end{itemize}
\LstRef
\end{frame}

\defverbatim[colored]\LstGen{%
\begin{lstlisting}[tabsize=8,showtabs,frame=single,language=make]
%.o : %.f90
	ifort -c -o $@ $<

foo : foo.o bar.o
	ifort -o $@ $^
\end{lstlisting}}

\begin{frame}
    \frametitle{Automatic Rules}
    Often, the rule is identical for many if not all objects, so we can generalise:
\LstGen
\end{frame}

\defverbatim[colored]\LstVars{%
\begin{lstlisting}[tabsize=8,showtabs,frame=single,language=make]
FC = ifort
FFLAGS = -warn all -check all
LD = $(FC)
LDFLAGS = $(FFLAGS) -l foo

%.o : %.f90
	$(FC) $(FFLAGS) -o $@ $<

foo : foo.o bar.o
	$(LD) $(LDFLAGS) -o $@ $^
\end{lstlisting}}

\begin{frame}
    \frametitle{Automatic Rules}
    Variables can be used to keep simplity it further. In the example below, it would be much easier to change between compilers:
\LstVars
\end{frame}

\begin{frame}
    \frametitle{Other Build Systems}
    \textbf{make} is used very widely. 
    
    That said, there are several other build systems out there.
    Wikipedia currently lists about 3 dozen of them.

\end{frame}

\begin{frame}
    \frametitle{Questions}
    This concludes this short chapter on Build Automation.

    Any questions so far?
\end{frame}
