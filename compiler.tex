\section{Compiler}
\begin{frame}
    \frametitle{Compiler}
    The main Fortran Compilers used in this centre are:
    \begin{itemize}
        \item Intel Fortran Compiler
        \item gfortran
    \end{itemize}
    Intel Fortran Compiler is better, gfortran is part of the GNU Compiler Collection and is free.

    On NCI, \texttt{mpif90} and \texttt{mpifort} are used as wrappers for Intel Fortran. 
\end{frame}
\begin{frame}[fragile]
    \frametitle{Compiler Warnings}
    \begin{lstlisting}[language=bash]
    -warn all           (ifort)
    -Wall               (gfortran)
    \end{lstlisting}
    Warnings are unexpected behaviour, e.g. using of uninitialised variables, unreachable code.
\end{frame}
\begin{frame}[fragile]
    \frametitle{Checks}
    \begin{lstlisting}[language=bash]
    -check all          (ifort)    
    -fcheck=all         (gfortran)
    \end{lstlisting}
    Run time checks slow the execution down, but useful during development.
    Checks include array bounds, but also give run-time warnings when temporary arrays are created.
\end{frame}
\begin{frame}[fragile]
    \frametitle{Optimisation Levels}
    \begin{lstlisting}[language=bash]
    -O<n>
    \end{lstlisting}
    \texttt{n} between 0 and 3.
    \begin{itemize}
    \item \texttt{-O0} means no optimisation, everything exactly as written
    \item \texttt{-O2} normal optimisation, usually safe
    \item \texttt{-O3} strong optimisation, can introduce errors, use with care.
    \end{itemize}
\end{frame}
\begin{frame}[fragile]
    \frametitle{Debugging Symbols}
    \begin{lstlisting}[language=bash]
    -g
    \end{lstlisting}
    Include actual code in executable. Can be read by debuggers or \texttt{objdump -S}
\end{frame}
\begin{frame}[fragile]
    \frametitle{Includes}
    \begin{lstlisting}[language=bash]
    -I <CPATH>
    -i <MOD_FILE>
    \end{lstlisting}
    \texttt{-i} and \texttt{-I} are used for compilation. They give the file (\texttt{-i}) and directory (\texttt{-I}) of the include and \texttt{.mod} files.

    \begin{lstlisting}[numbers=none]
    -I ./includes -l foo
    \end{lstlisting}
    searches for the include file \texttt{foo.mod} in \texttt{./includes} first, then all directories in \texttt{\$CPATH}.
\end{frame}
\begin{frame}[fragile]
    \frametitle{Linking to Libraries}
    \begin{lstlisting}[language=bash]
    -L <LIB_PATH>
    -l <LIB_NAME>
    \end{lstlisting}
    \texttt{-l} and \texttt{-L} are used for linking. They give the directory (\texttt{-L}) and library name (\texttt{-l}). 

    \begin{lstlisting}[language=bash]
    -L ./libs -l foo
    \end{lstlisting}
    searches for the library files \texttt{libfoo.a} or \texttt{libfoo.so} in \texttt{./libs} first, then all directories in \texttt{\$LD\_LIBRARY\_PATH}.
\end{frame}

\begin{frame}
    \frametitle{Questions}
    This concludes this short chapter on compiler options.

    Any questions so far?
\end{frame}
