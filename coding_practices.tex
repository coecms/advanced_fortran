\section{Coding Practices}

% \begin{frame}
%     \frametitle{Comments}
%     We all use comments too sparingly.
%     
%     Someone else, or you in 2 years, will not remember what was in your head when you wrote this piece of code.
%     So when writing code, keep readability in mind. 
% 
% \end{frame}
% 
% \begin{frame}[fragile]
%     \frametitle{Comments continued}
% 
%     Comments should convey why you did what you did, not what you did.
%     \begin{lstlisting}
%     ! increment i
%     i = i + 1
%     \end{lstlisting}
%     is useless
% 
%     \begin{lstlisting}
%     ! Calc absolute distance between A and B
%     dist = sqrt((A(1) - B(1))**2 + (A(2) - B(2))**2)
%     \end{lstlisting}
%     is good.
% 
% \end{frame}
% 
\begin{frame}
    \frametitle{Understandable Code}
    While writing code, you are actively thinking about writing this code.
    Every line that you write will make perfect sense while you're writing it.

    But when you look at someone else's code, or even your own code from 6 months ago,
    you won't have those same thoughts in your head any more.

    And then you will have to spend a lot of brain power to figure out what the code is doing,
    and why it's doing what it's doing.

    For your own sanity: \textbf{Keep your code understandable}
\end{frame}

\begin{frame}[fragile]
    \frametitle{Comments}
    The easiest way to keep your code understandable is to write comments.
    In general: The more the merrier, but:

    Comments should explain \textbf{why} you're doing what you're doing,
    rather than what you're doing:

    \begin{lstlisting}[language=Fortran]
    ! Increment i
    i = i + 1
    \end{lstlisting}
    This comment is not useful.
\end{frame}

\begin{frame}
    \frametitle{Naming}
    Often you will think up names. 
    Names for procedures, variables.

    Think about what to call it. 
    Make sure that the name really represents what is stored in the variable.

    It's okay to have an iterator called \texttt{i}, if you're consistent about it.
    
    But if you want to store some user name, please call it something like \texttt{cUserName}, and not just \texttt{u}.
    (The leading \texttt{c} stands for the type of variable: \texttt{CHARACTER}.)
\end{frame}

\begin{frame}[fragile]
    \frametitle{Magic Numbers}
    Sometimes you need some numbers in your code, for example to check whether another number is even:
    \begin{lstlisting}
    if (mod(n, 2) == 0) then
    \end{lstlisting}

    But sometimes it's more confusing:
    \begin{lstlisting}
    if (iErrorCode == 264) then
    \end{lstlisting}
    What's the \texttt{264}? What do I do here? This \texttt{264} is what programmers call a 'magic number' --
    and it hinders understanding the code.
\end{frame}

\begin{frame}[fragile]
    \frametitle{Avoiding Magic Numbers}
    The way to avoid magic numbers is to create constants.
    Say, this \texttt{264} was the error code for \textbf{File Not Found}.

    Then you could do something like this:
    \begin{lstlisting}
        INTEGER, PARAMETER :: FileNotFoundError = 264
        ...
        if (iErrorCode == FileNotFoundError) then
    \end{lstlisting}
    Far easier to read.
\end{frame}

\begin{frame}
    \frametitle{Error Checking/Handling}
    Many Fortran routines that could conceivably fail have 
    optional error arguments.

    Examples:

    \begin{itemize}
        \item \texttt{OPEN}, \texttt{READ}, \texttt{WRITE}, even \texttt{CLOSE} have an optional \texttt{IOSTAT} argument that returns an \texttt{INTEGER}.
        \item \texttt{ALLOCATE}, \texttt{DEALLOCATE} have \texttt{STAT}.
    \end{itemize}

    By convention, 0 means that there was no issue. A negative value refers to a warning, and a positive value corresponds to an error.
    Think about what might happen, and how the program should react to this.

    Use them.
\end{frame}

\begin{frame}[fragile]
    \frametitle{Error passing}
    Two main error passing philosophies:

    In one, the error is an (often optional) \texttt{INTENT(OUT)} parameter. 
    \begin{lstlisting}
    CALL mySub(arg1, arg2, error)
    \end{lstlisting}

    In the other, you call a function that returns the error code:
    \begin{lstlisting}
    error = myFunc(arg1, arg2)
    \end{lstlisting}

\end{frame}

\begin{frame}
    \frametitle{Fragile and Robust}
    Error handling allows you to design your software to be either fragile or robust:

    A \textbf{robust} program will attempt to plow ahead, possibly logging the error. 
    While desirable for a deployed program, this might obscure the real cause of odd behaviour later on.

    A \textbf{fragile} program will fall over early upon encountering an unexpected situation. 
    This is particularly useful during programming and debugging, but can be annoying in production.
\end{frame}
