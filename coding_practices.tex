\section{Coding Practices}

% \begin{frame}
%     \frametitle{Comments}
%     We all use comments too sparingly.
%     
%     Someone else, or you in 2 years, will not remember what was in your head when you wrote this piece of code.
%     So when writing code, keep readability in mind. 
% 
% \end{frame}
% 
% \begin{frame}[fragile]
%     \frametitle{Comments continued}
% 
%     Comments should convey why you did what you did, not what you did.
%     \begin{lstlisting}
%     ! increment i
%     i = i + 1
%     \end{lstlisting}
%     is useless
% 
%     \begin{lstlisting}
%     ! Calc absolute distance between A and B
%     dist = sqrt((A(1) - B(1))**2 + (A(2) - B(2))**2)
%     \end{lstlisting}
%     is good.
% 
% \end{frame}
% 

\begin{frame}
    \frametitle{Error Checking/Handling}
    Many Fortran routines that could conceivably fail have 
    optional error arguments.

    Examples:

    \begin{itemize}
        \item \texttt{OPEN}, \texttt{READ}, \texttt{WRITE}, even \texttt{CLOSE} have an optional \texttt{IOSTAT} argument that returns an \texttt{INTEGER}.
        \item \texttt{ALLOCATE}, \texttt{DEALLOCATE} have \texttt{STAT}.
    \end{itemize}

    By convention, 0 means that there was no issue. A negative value refers to a warning, and a positive value corresponds to an error.
    Think about what might happen, and how the program should react to this.

    Use them.
\end{frame}

\begin{frame}[fragile]
    \frametitle{Error passing}
    Two main error passing philosophies:

    In one, the error is an (often optional) \texttt{INTENT(OUT)} parameter. 
    \begin{lstlisting}
    CALL mySub(arg1, arg2, error)
    \end{lstlisting}

    In the other, you call a function that returns the error code:
    \begin{lstlisting}
    error = myFunc(arg1, arg2)
    \end{lstlisting}

\end{frame}

\begin{frame}
    \frametitle{Fragile and Robust}
    Error handling allows you to design your software to be either fragile or robust:

    A \textbf{robust} program will attempt to plow ahead, possibly logging the error. 
    While desirable for a deployed program, this might obscure the real cause of odd behaviour later on.

    A \textbf{fragile} program will fall over early upon encountering an unexpected situation. 
    This is particularly useful during programming and debugging, but can be annoying in production.
\end{frame}
